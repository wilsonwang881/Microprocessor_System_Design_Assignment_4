\documentclass[11pt,letterpaper,titlepage]{article}

\usepackage{geometry}
\geometry{left=2cm,right=2cm,top=2cm,bottom=3cm}

\usepackage{setspace}
\onehalfspacing

\usepackage{multicol}
\setlength{\columnsep}{3em}

\usepackage{booktabs}

\usepackage[table,x11names]{xcolor}

\usepackage{multirow}

\usepackage{pgfgantt}

\usepackage{listings}

\usepackage{xcolor}
\definecolor{vgreen}{RGB}{104,180,104}
\definecolor{vblue}{RGB}{49,49,255}
\definecolor{vorange}{RGB}{255,143,102}

\lstdefinestyle{C-style}
{
    language=C,
    basicstyle=\small\ttfamily,
    keywordstyle=\color{vblue},
    identifierstyle=\color{black},
    commentstyle=\color{vgreen},
    breaklines=true,
    % numbers=left,
    numberstyle=\tiny\color{black},
    numbersep=11pt,
    tabsize=4,
    moredelim=*[s][\colorIndex]{[}{]},
    literate=*{:}{:}1
}

\lstdefinestyle{txt-style}
{
    basicstyle=\small\ttfamily,
    breaklines=true,
    % numbers=left,
    numbersep=11pt,
    tabsize=4,
    moredelim=*[s][\colorIndex]{[}{]},
    literate=*{:}{:}1
}

\usepackage{tikz}
\usetikzlibrary{shapes.geometric, arrows, positioning, fit,calc}
\newcommand*\circled[1]{\tikz[baseline=(char.base)]{
            \node[shape=circle,draw,inner sep=1pt] (char) {#1};}}
            
\usepackage{hyperref}
\hypersetup{
    colorlinks,
    citecolor=black,
    filecolor=black,
    linkcolor=black,
    urlcolor=black
}

\usepackage{pifont}

\usepackage[toc,page]{appendix}

\pagestyle{empty}
\usepackage{tikz}
\usetikzlibrary{shapes.geometric, arrows}

\usetikzlibrary{mindmap,trees}
\usepackage{verbatim}

\usepackage{indentfirst}
\setlength{\parindent}{2em}

\usepackage{listings}

\usepackage{chngcntr}
\counterwithin{section}{part}
\renewcommand\thesection{\arabic{section}}

\usepackage{graphicx}

\usepackage{subcaption}

\usepackage{fancyhdr}

\pagestyle{fancy}
\lhead{}
\rhead{}
\lfoot{ECEN 749 Section 601 Assignment 4}
\cfoot{\thepage}
\rfoot{@Lei Wang (Wilson)}
\renewcommand{\headrulewidth}{0pt}
\renewcommand{\headwidth}{\textwidth}
\renewcommand{\footrulewidth}{0.4pt}
\newcommand{\RomanNumeralCaps}[1]
    {\MakeUppercase{\romannumeral #1}}

\makeatletter
\newcommand*\@lbracket{[}
\newcommand*\@rbracket{]}
\newcommand*\@colon{:}
\newcommand*\colorIndex{%
    \edef\@temp{\the\lst@token}%
    \ifx\@temp\@lbracket \color{black}%
    \else\ifx\@temp\@rbracket \color{black}%
    \else\ifx\@temp\@colon \color{black}%
    \else \color{vorange}%
    \fi\fi\fi
}
\makeatother

\usepackage{trace}

\begin{document}

\begin{titlepage}
  \centering
	{\scshape\large Texas A\&M University \par}
	\vspace{1cm}
	{\scshape\Large Department of Electrical and Computer Engineering \par}
	\vspace{4cm}
    \vspace{0.5cm}
	{\huge\bfseries ECEN 749 Microprocessor System Design\par}
	\vspace{4cm}
	{\Large Assignment 4 Report (Section 601)\par}
	\vspace{1cm}
	{\Large Student: Lei Wang (Wilson)\par}
	\vspace{1cm}
	{\Large UIN: 829009485\par}
	\vspace{1cm}
	{\Large Instructor: Dr. Paul V. Gratz\par}
	\vspace{4cm}
	\vfill

  % Bottom of the page
	{\large Submitted: April 28th, 2020 \par}

\end{titlepage}

\newpage

\tableofcontents{}

\newpage

\part{Introduction}

The assignment aims at teaching students to boot Ubuntu in a virtual machine environment and use that environment to compile a Linux operating system. The compiled Linux operating system is used to test the student-written module that implements multiplication in software because the inaccessibility to FPGA hardware that can be used to implement a hardware multiplier. The custom-written Linux module is tested as a loadable one and as a built-in one.

\part{Procedure}

\section{Boot Ubuntu}

\begin{enumerate}
    
    \item Download the ISO image of Ubuntu 18.04.4 LTS from \url{https://ubuntu.com/download/desktop}.
    
    \item Download Oracle VM VirtualBox from \url{https://www.virtualbox.org/wiki/Downloads} and install the software.
    
    \item Launch VirtualBox and select \textbf{Machine} $\rightarrow$ \textbf{New}. 
    
    \item In the dialog window, enter the following:
    
    \begin{table}[ht]
    \centering
    \begin{tabular}{@{}cc@{}}
    \toprule
    Name           & Ubuntu                          \\ \midrule
    Machine Folder & \textless{}custom\textgreater{} \\ \midrule
    Type           & Linux                           \\ \midrule
    Version        & Ubuntu (64-bit)                 \\ \bottomrule
    \end{tabular}
    \end{table}
    
    Click on \textbf{Next}.
    
    \item Set \textbf{Memory size} to 4096 MB. Click on \textbf{Next}.
    
    \item Select \textbf{Create a virtual hard disk now}. Click on \textbf{Create}.
    
    \item Select \textbf{VDI (VirtualBox Disk Image)}. Click on \textbf{Next}.
    
    \item Select \textbf{Fixed size}. Click on \textbf{Next}.
    
    \item Allocate a space of 30 GB. Click on \textbf{Create}. Wait for VirtualBox to finish executing.
    
    \item Select the virtual machine from the list. Open the settings. Go to \textbf{System} $\rightarrow$ \textbf{Processor}. Increase the processor count to 4 (adjust accordingly based on CPU specifications). Go to \textbf{Display} $\rightarrow$ \textbf{Screen}, increase \textbf{Video Memory} to 128 MB. Click \textbf{OK} to close the window.
    
    \item Click on \textbf{Start}. Select the downloaded Ubuntu ISO image as the start-up disk. Click on \textbf{Start}. Follow the on-screen prompts to install Ubuntu. A virtual machine restart is needed to finish the installation.
    
    \item Adjust the resolution if needed.
    
    \item In Ubuntu virtual machine, open a terminal and run the following command one by one:
    
    \begin{table}[ht]
    \begin{tabular}{l}
    \toprule
    \texttt{sudo apt-get update}                                                                  \\ \midrule
    \texttt{sudo apt-get install git fakeroot build-essential ncurses-dev xz-utils libssl-dev bc} \\ \midrule
    \texttt{sudo apt-get install vim}                                                             \\ \midrule
    \texttt{sudo apt-get install vim-gtk3}                                                        \\ \midrule
    \texttt{sudo apt-get install make}                                                            \\ \midrule
    \texttt{sudo apt-get install gcc}                                                             \\ \midrule
    \texttt{sudo apt-get install bison}                                                           \\ \midrule
    \texttt{sudo apt-get install flex}                                                            \\ \midrule
    \texttt{sudo apt-get install libelf-dev}                                                      \\ \bottomrule           
    \end{tabular}
    \caption{Commands to setup the Linux development environment}
    \end{table}
    
\end{enumerate}

\section{Compile Linux-4.20.17}

\begin{enumerate}
    
    \item Open a terminal. Download the Linux zip file by running:
    
    \texttt{wget https://www.kernel.org/pub/linux/kernel/v4.x/linux-4.20.17.tar.gz}
    
    \item Unzip the file by running:
    
    \texttt{tar xf linux-4.20.17.tar.gz}
    
    \item Change to the unzipped Linux directory by running:
    
    \texttt{cd linux-4.20.17}
    
    \item Load the kernel compilation configuration by running:
    
    \texttt{make ARCH=x86 menuconfig}
    
    \item Remove \textbf{multimedia} and \textbf{soundcard} inside the \textbf{Device Driver} category. Move the cursor to the corresponding entry and press the space key until the entry starts as $<>$. Save the configurations and exit to the terminal.
    
    \item Compile and install Linux-4.20.17 by running (note the last number in each command represents the number of CPU cores allocated to the virtual machine):
    
    \texttt{sudo make ARCH=x86 -j 4}
    
    \texttt{sudo make ARCH=x86 modules\_install -j 4}
    
    \texttt{sudo make ARCH=x86 install -j 4}
    
\end{enumerate}

\section{Create the Kernel Module}

\begin{enumerate}
    
    \item Open a new terminal. Create a directory named \textbf{multiplier} by running:
    
    \texttt{mkdir multiplier}
    
    \item In the directory, create the following files: \texttt{Makefile}, \texttt{multiplier.h}, \texttt{multiplier.c}, \texttt{devtest.c}. Copy the content from the appendix into the corresponding files.
    
    \item Compile the \textbf{multiplier} module by running:
    
    \texttt{make}
    
    \item Compile the test program \textbf{devtest.c} by running:
    
    \texttt{gcc devtest.c -o devtest}
    
\end{enumerate}

\section{Boot the Compiled Linux in VirtualBox}

\begin{enumerate}
    
    \item In Ubuntu, reboot the system. Hold the shift key.
    
    \item The GRUB bootloader menu should show up. Select \textbf{Advanced options for Ubuntu} $\rightarrow$ \textbf{Ubuntu, with Linux 4.20.17}
    
    \item Check the Linux version by running:
    
    \texttt{uname -r}
    
\end{enumerate}

\section{Insert the Kernel Module after Booting Linux}

\begin{enumerate}
    
    \item Open a terminal and change to the directory to the one that contains the compiled kernel module. Install the module by running:
    
    \texttt{sudo insmod multiplier.ko}
    
    \item Confirm the \textbf{multiplier} module is installed by checking the output after running:
    
    \texttt{lsmod | grep multiplier}
    
    \item Check the output when installing the module by running:
    
    \texttt{dmesg | tail}
    
    \item Create the device file by running (replace 240 with the major number indicated in the previous console output): 
    
    \texttt{sudo mknod /dev/multiplier c 240 0}
    
    \item Change the permission of the device file by running:
    
    \texttt{sudo chmod 666 /dev/multiplier}
    
    \item Run the device driver test:
    
    \texttt{./devtest}
    
    \item Remove the \textbf{multiplier} module and check whether it has been removed:
    
    \texttt{sudo rm /dev/multiplier} 
    
    \texttt{sudo rmmod multiplier}
    
    \texttt{lsmod | grep multiplier}
    
\end{enumerate}

\section{Built-in Kernel Module and Test}

\begin{enumerate}
    
    \item Boot into the default Ubuntu version, not the compiled 4.20.17 Linux version.
    
    \item In the \textbf{driver} directory located in the unzipped Linux directory, create a new directory called \textbf{multiplier\_module}. Place \texttt{multiplier.c} and \texttt{multiplier.h} in the created directory.
    
    \item In the same directory, create a file \texttt{Makefile} with the following content:
    
    \texttt{obj-\$(CONFIG\_MULTIPLIER\_DRIVER) += multiplier.o}
    
    \item In the same directory, create a file \texttt{Kconfig} file with the following content:
    
    \texttt{config MULTIPLIER\_DRIVER}
    
    \texttt{tristate "multiplier\_driver"}
    
    \texttt{depends on X86}
    
    \texttt{default y if X86}
    
    \texttt{help}
    
    \texttt{ refer to ECEN449@TAMU}
    
    \item Add the following lines to \texttt{Makefile} in the \textbf{driver} directory:
    
    \texttt{\# ECEN 449}
    
    \texttt{obj-\$(CONFIG\_MULTIPLIER\_DRIVER) += multiplier\_driver/}
    
    \item Add the following line before ``\texttt{endmenu}'' in \texttt{Kconfig} in the \textbf{driver} directory:
    
    \texttt{source "drivers/multiplier\_driver/Kconfig"}
    
    \item In the Linux directory, run:
    
    \texttt{make ARCH=x86 menuconfig}
    
    \item In the compilation configuration menu, go to \textbf{Device Driver} $\rightarrow$ \textbf{multiplier\_driver}. Press the space key until the \textbf{multiplier\_driver} entry starts with $<*>$.
    
    \item Now the multiplier module has been added to the compilation configuration. Re-compile and install the new kernel by running:
    
    \texttt{sudo make clean}
    
    \texttt{sudo make ARCH=x86 -j 4}
    
    \texttt{sudo make ARCH=x86 modules\_install -j 4}
    
    \texttt{sudo make ARCH=x86 install -j 4}
    
    \item Reboot the Ubuntu virtual machine and select the re-compiled kernel from the GRUB menu.
    
    \item Check the loading information and the major number by running:
    
    \texttt{dmesg | grep multiplier}
    
    \item Create the device file and change its permission by running:
    
    \texttt{sudo mknod /dev/multiplier c <Major Number> 0}
    
    \texttt{sudo chmod 666 /dev/multiplier}
    
    \item Run the device driver test.
    
\end{enumerate}

\newpage

\part{Results}

\section{Compilation of Linux 4.20.17}

\section{}

\newpage

\part{Conclusion}

\textbf{Q: What are the advantages and disadvantages of both the loadable kernel modules and
built-in module approach?}

A:

\textbf{Q: Describe both one good thing and one thing that can be improved about your experience
in the labs this semester.}

A: 

\newpage

\begin{appendices}

\section{Makefile}
\label{appendix:makefile}
\lstinputlisting[style={txt-style}]{Makefile}

\section{multiplier.h}
\label{appendix:sourcecode_multiplier_header}
\lstinputlisting[style={txt-style}]{multiplier.h}

\section{multiplier.c}
\label{appendix:sourcecode_multiplier}
\lstinputlisting[style={txt-style}]{multiplier.c}

\section{devtest.c}
\label{appendix:sourcecode_devtest}
\lstinputlisting[style={txt-style}]{devtest.c}

\end{appendices}

\end{document}
